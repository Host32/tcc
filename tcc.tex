%% abtex2-modelo-relatorio-tecnico.tex, v<VERSION> laurocesar
%% Copyright 2012-<COPYRIGHT_YEAR> by abnTeX2 group at http://www.abntex.net.br/ 
%%
%% This work may be distributed and/or modified under the
%% conditions of the LaTeX Project Public License, either version 1.3
%% of this license or (at your option) any later version.
%% The latest version of this license is in
%%   http://www.latex-project.org/lppl.txt
%% and version 1.3 or later is part of all distributions of LaTeX
%% version 2005/12/01 or later.
%%
%% This work has the LPPL maintenance status `maintained'.
%% 
%% The Current Maintainer of this work is the abnTeX2 team, led
%% by Lauro César Araujo. Further information are available on 
%% http://www.abntex.net.br/
%%
%% This work consists of the files abntex2-modelo-relatorio-tecnico.tex,
%% abntex2-modelo-include-comandos and references.bib
%%

% ------------------------------------------------------------------------
% ------------------------------------------------------------------------
% abnTeX2: Modelo de Relatório Técnico/Acadêmico em conformidade com 
% ABNT NBR 10719:2015 Informação e documentação - Relatório técnico e/ou
% científico - Apresentação
% ------------------------------------------------------------------------ 
% ------------------------------------------------------------------------

\documentclass[
	% -- opções da classe memoir --
	12pt,				% tamanho da fonte
	openright,			% capítulos começam em pág ímpar (insere página vazia caso preciso)
	twoside,			% para impressão em recto e verso. Oposto a oneside
	a4paper,			% tamanho do papel. 
	% -- opções da classe abntex2 --
	%chapter=TITLE,		% títulos de capítulos convertidos em letras maiúsculas
	%section=TITLE,		% títulos de seções convertidos em letras maiúsculas
	%subsection=TITLE,	% títulos de subseções convertidos em letras maiúsculas
	%subsubsection=TITLE,% títulos de subsubseções convertidos em letras maiúsculas
	% -- opções do pacote babel --
	english,			% idioma adicional para hifenização
	french,				% idioma adicional para hifenização
	spanish,			% idioma adicional para hifenização
	brazil,				% o último idioma é o principal do documento
	]{abntex2}


% ---
% PACOTES
% ---

% ---
% Pacotes fundamentais 
% ---
\usepackage{lmodern}			% Usa a fonte Latin Modern
\usepackage[T1]{fontenc}		% Selecao de codigos de fonte.
\usepackage[utf8]{inputenc}		% Codificacao do documento (conversão automática dos acentos)
\usepackage{indentfirst}		% Indenta o primeiro parágrafo de cada seção.
\usepackage{color}				% Controle das cores
\usepackage{graphicx}			% Inclusão de gráficos
\usepackage{microtype} 			% para melhorias de justificação
% ---

% ---
% Pacotes adicionais, usados no anexo do modelo de folha de identificação
% ---
\usepackage{multicol}
\usepackage{multirow}
% ---
	
% ---
% Pacotes adicionais, usados apenas no âmbito do Modelo Canônico do abnteX2
% ---
\usepackage{lipsum}				% para geração de dummy text
% ---

% ---
% Pacotes de citações
% ---
\usepackage[brazilian,hyperpageref]{backref}	 % Paginas com as citações na bibl
\usepackage[alf]{abntex2cite}	% Citações padrão ABNT

% ---
% Pacotes de check symbol
% ---
\usepackage{tikz}
\def\checkmark{\tikz\fill[scale=0.4](0,.35) -- (.25,0) -- (1,.7) -- (.25,.15) -- cycle;}

% --- 
% CONFIGURAÇÕES DE PACOTES
% --- 

% ---
% Configurações do pacote backref
% Usado sem a opção hyperpageref de backref
\renewcommand{\backrefpagesname}{Citado na(s) página(s):~}
% Texto padrão antes do número das páginas
\renewcommand{\backref}{}
% Define os textos da citação
\renewcommand*{\backrefalt}[4]{
	\ifcase #1 %
		Nenhuma citação no texto.%
	\or
		Citado na página #2.%
	\else
		Citado #1 vezes nas páginas #2.%
	\fi}%
% ---

% ---
% Informações de dados para CAPA e FOLHA DE ROSTO
% ---
\titulo{Projeto de Automação de Operações e Entrega Contínua para Empresa de Desenvolvimento de Software.}
\autor{Ivan de Moura Miranda}
\local{Conselheiro Lafaiete}
\data{2016}
\instituicao{%
  Faculdade Presidente Antônio Carlos -- FUPAC
  \par
  Engenharia de Computação
  \par
  Prof. Me. Jean Carlo Mendes}
\tipotrabalho{Relatório técnico}
% O preambulo deve conter o tipo do trabalho, o objetivo, 
% o nome da instituição e a área de concentração 
\preambulo{Relatório técnico apresentado como trabalho de conclusão de curso para obtenção do diploma do curso de Engenharia de Computação, Faculdade Presidente Antônio Carlos de Conselheiro Lafaiete. }
% ---

% ---
% Configurações de aparência do PDF final

% alterando o aspecto da cor azul
\definecolor{blue}{RGB}{41,5,195}

% informações do PDF
\makeatletter
\hypersetup{
     	%pagebackref=true,
		pdftitle={\@title}, 
		pdfauthor={\@author},
    	pdfsubject={\imprimirpreambulo},
	    pdfcreator={LaTeX with abnTeX2},
		pdfkeywords={abnt}{latex}{abntex}{abntex2}{relatório técnico}, 
		colorlinks=true,       		% false: boxed links; true: colored links
    	linkcolor=blue,          	% color of internal links
    	citecolor=blue,        		% color of links to bibliography
    	filecolor=magenta,      		% color of file links
		urlcolor=blue,
		bookmarksdepth=4
}
\makeatother
% --- 

% --- 
% Espaçamentos entre linhas e parágrafos 
% --- 

% O tamanho do parágrafo é dado por:
\setlength{\parindent}{1.3cm}

% Controle do espaçamento entre um parágrafo e outro:
\setlength{\parskip}{0.2cm}  % tente também \onelineskip

% ---
% compila o indice
% ---
\makeindex
% ---

% ----
% Início do documento
% ----
\begin{document}

% Seleciona o idioma do documento (conforme pacotes do babel)
%\selectlanguage{english}
\selectlanguage{brazil}

% Retira espaço extra obsoleto entre as frases.
\frenchspacing 

% ----------------------------------------------------------
% ELEMENTOS PRÉ-TEXTUAIS
% ----------------------------------------------------------
% \pretextual

% ---
% Capa
% ---
\imprimircapa
% ---

% ---
% Folha de rosto
% (o * indica que haverá a ficha bibliográfica)
% ---
\imprimirfolhaderosto*
% ---


% ---
% RESUMO
% ---

% resumo na língua vernácula (obrigatório)
\setlength{\absparsep}{18pt} % ajusta o espaçamento dos parágrafos do resumo
\begin{resumo}
 Este trabalho descreve um projeto de otimização do processo de entrega de uma empresa de desenvolvimento de softwares. O objetivo é identificar as etapas de baixa produtividade para implementar ferramentas e aplicar metodologias que melhoram a qualidade dos softwares desenvolvidos. Durante o processo de desenvolvimento diversas tarefas repetitivas são executadas manualmente, com a automatização destas conseguimos reduzir o tempo investido no projeto, e com a aplicação de boas práticas de desenvolvimento, também garantimos melhor qualidade dos produtos desenvolvidos.

 \noindent
 \textbf{Palavras-chaves}: entrega contínua, integração contínua, \textit{devops}, metodologias ágeis.
\end{resumo}
% ---

% ---
% inserir lista de ilustrações
% ---
%\pdfbookmark[0]{\listfigurename}{lof}
%\listoffigures*
%\cleardoublepage
% ---

% ---
% inserir lista de tabelas
% ---
\pdfbookmark[0]{\listtablename}{lot}
\listoftables*
% ---

% ---
% inserir lista de abreviaturas e siglas
% ---
\begin{siglas}
	\item[AWS] \textit{Amazon Web Services}
	\item[BaaS] \textit{Back-end as a Service}
	\item[bug] \textit{Defeito, falha ou erro no código de um programa que provoca seu mau funcionamento.}
	\item[deploy] \textit{Implantação de uma versão do software em ambiente de produção}
	\item[devops] \textit{Development and Operations}
	\item[IaaS] \textit{Infrastructure as a Service}
	\item[PaaS] \textit{Platform as a Service}
	\item[SaaS] \textit{Software as a Service}
	\item[SSH] \textit{Secure Shell}
\end{siglas}
% ---

% ---
% inserir lista de símbolos
% ---
%\begin{simbolos}
%  \item[$ \Gamma $] Letra grega Gama
%  \item[$ \Lambda $] Lambda
%  \item[$ \zeta $] Letra grega minúscula zeta
%  \item[$ \in $] Pertence
%\end{simbolos}
% ---

% ---
% inserir o sumario
% ---
\pdfbookmark[0]{\contentsname}{toc}
\tableofcontents*
\cleardoublepage
% ---


% ----------------------------------------------------------
% ELEMENTOS TEXTUAIS
% ----------------------------------------------------------
\textual

\chapter{Introdução}
Segundo uma pesquisa publicada pelo \citeonline{StackOverflow:DeveloperSurvey2016}, dentre os maiores desafios encontrados por desenvolvedores em seu ambiente de trabalho vemos como mais influentes a documentação ruim, os requisitos mal especificados e o processo de desenvolvimento ineficiente.

Diante dessa realidade percebemos que o ambiente de desenvolvimento de softwares é negativamente afetado pela ineficiência do processo que muitas das vezes é excessivamente burocrático e pouco produtivo.

 \begin{citacao}
Mas ser burocrático não é o problema, pois as regras são necessárias para nortear qualquer tipo de trabalho. O problema reside, é claro, em ser excessivamente burocrático, gerando atividades/produtos que não agregam valor e consomem desnecessariamente tempo da equipe. \cite{EBusiness:ManifestoAgil}
\end{citacao}

Em busca de resolver estes e outros problemas, várias metodologias foram criadas, como os métodos ágeis, a cultura \textit{devops}, e as ferramentas que objetivam a automação dos processos de operação, permitindo a entrega contínua de software com a mínima intervenção humana, em busca de agilidade e qualidade do sistema entregue. O problema é que essa automação exige conhecimento de muitas tecnologias e ferramentas diferentes, o que é visto na maioria das vezes como barreira por desenvolvedores que não tem familiaridade com especificações de infraestrutura.

De outro lado, várias empresas viram uma oportunidade de negócio atrelada a resolução destes problemas, e diversas plataformas começaram a surgir nos últimos anos, como o \textit{Developer Cloud Service} \cite{Oracle:DevOps}, \textit{Open Shift} \cite{RedHat:OpenShift} e o \textit{AWS Lambda} \cite{Amazon:Lambda}. Estas soluções buscam facilitar a vida dos desenvolvedores oferecendo infraestrutura, plataforma ou \textit{back-end} como serviço para as empresas de desenvolvimento, e algumas ainda oferecem ferramentas para gestão dos projetos do time de desenvolvimento.

Apesar destas soluções prontas, o fato curioso é que, se não levarmos em conta o hardware e considerarmos apenas as ferramentas de automação fornecidas, boa parte delas são gratuitas e de código fonte aberto como o Git, Jenkins e Kubernetes. O próprio Open Shift por exemplo possui uma versão de código aberta que pode ser instalado na sua própria empresa com acesso a todos os recursos. Então a vantagem em contratar esse tipo de serviço se deve apenas a hospedagem na nuvem e ao baixo custo de manutenção dessas ferramentas que dispensam a necessidade de um profissional especializado contratado pela sua empresa, porém este custo ainda não é totalmente viável para micro e pequenas empresas, que de acordo com uma pesquisa realizada pela \citeonline{ABES:Pesquisa} representam 95,1\% das empresas de desenvolvimento e produção de tecnologia no Brasil.

Diante desse cenário, boa parte do mercado nacional pode se sentir desmotivado a contratar um serviço desse tipo, pois é muito difícil prever o benefício real proporcionado, afinal, o custo do investimento é fácil calcular, mas o lucro estimado é complicado, uma vez que eficiência e eficácia precisam ser convertidas em dinheiro baseadas na confiança de que a automação irá realmente resolver os problemas da empresa.

Uma alternativa a este investimento ainda existe, que é a capacitação profissional do time de desenvolvimento para a utilização das ferramentas gratuitas em servidores locais, que não precisam ser computadores muito potentes devido a baixa necessidade de escalabilidade e performance, uma vez que os usuários desses sistemas serão os próprios colaboradores da empresa.

Para implantação desse ambiente automatizado que utiliza apenas ferramentas gratuitas, é necessário o estudo de uma gama de soluções a fim de identificar a que mais atende a real necessidade da empresa.

% ---
% Capitulos de Objetivo Geral
% ---
\section{Objetivos}

Em uma tentativa de validar os benefícios da automação e entrega contínua num cenário real, graças ao apoio da DevelOP, uma empresa de desenvolvimento de software situada em Ouro Preto, este trabalho se propõe a implementar um ambiente de desenvolvimento integrado baseado no estudo de boas práticas e ferramentas gratuitas existentes no mercado buscando a automação das tarefas de operações e o aumento da qualidade dos softwares desenvolvidos. Objetiva-se também avaliar a eficácia dos métodos propostos graças a uma analise da performance da empresa antes e depois da implementação do projeto.

% ---
% Capitulos de Objetivos Específicos
% ---
\section{Organização dos capítulos}

% ----------------------------------------------------------
% PARTE - preparação da pesquisa
% ----------------------------------------------------------
\part{Preparação do relatório}

% ---
% Capitulos de Metodologia
% ---
\chapter{Metodologia}

A fim de avaliar os benefícios das soluções propostas, este trabalho foi fundamentado na investigação a respeito do tema. As situações referentes ao objeto de estudo, que no caso se trata do processo de desenvolvimento de softwares da empresa DevelOP, foram examinadas com olhar investigativo buscando atingir a maior veracidade possível.

O projeto aborda conhecimentos a respeito da gerência de projetos, desenvolvimento e entrega de softwares e para isso se fez necessário direcionar a abordagem com base na utilização de material teórico, estabelecendo uma linha de investigação pela qual foi conduzido o trabalho para levantar todo o material necessário com o intuito de estabelecer uma avaliação prática dos resultados obtidos.

O projeto foi divido em 3 etapas. A primeira se baseou no estudo do cenário atual da empresa coletando e analisando dados em busca de identificar pontos problemáticos e levantar propostas de soluções. A segunda fase teve por objetivo implementar as soluções propostas e analisar seus efeitos no ambiente da empresa. A terceira fase teve a responsabilidade de mostrar os resultados e avaliar as soluções oferecidas e as vantagens que propõe.

Este trabalho atuou utilizando o método experimental de forma a estudar os fatores que influenciam o tempo e a qualidade do processo de desenvolvimento, desta forma avaliando a veracidade das hipóteses levantadas.

\section{Procedimentos para coleta e análise dos dados}

A coleta dos dados foi feita através do acompanhamento das rotinas da empresa para que o conhecimento da área no qual está focado o tema seja ampliado e permitindo uma análise mais enfática quando se necessita recolher informações de variados aspectos relacionados, assim preenchendo melhor as lacunas no projeto.

A análise dos dados foi feita através do comparativo entre o antes e depois da implementação das soluções propostas. Comparando o tempo necessário para entregar uma nova versão do software, a relação entre a quantidade \textit{builds} quebradas e bem sucedidas por dia, a quantidade de  \textit{bugs} reportados por semana, a facilidade de manutenção do código e a satisfação dos desenvolvedores com o processo de desenvolvimento.


% ---
% Capitulos de desenvolvimento
% ---
\chapter{Análise do cenário atual}

\section{Associação de Desenvolvimento de Software, Produtos e Pessoas da Região dos Inconfidentes Mineiros - DevelOP}

A DevelOP é uma associação privada fundada em 2016 por um grupo de profissionais em tecnologia da informação que desenvolve softwares, produtos e pessoas na região dos inconfidentes mineiros. (Pegar mais informações sobre a fundação com o Álvaro)

Situada no centro de Ouro Preto, a empresa tem como principal atividade econômica o desenvolvimento de programas de computador sob encomenda, mas também atua em atividades de desenvolvimento e licenciamento de programas de computador customizável, consultoria em tecnologia da informação e atividades de apoio à educação.

(Visão e Missão.)

Apesar de ser uma empresa nova no mercado, a DevelOP já possui alguns projetos em seu portfólio, como o evento Empreenda. Em Ação! realizado em julho de 2016 onde equipes formadas por alunos de disciplinas de Empreendedorismo da Universidade Federal de Ouro Preto (UFOP) competem na idealização, planejamento e apresentação de modelos de negócio reais e o atual projeto em desenvolvimento, onde a empresa presta serviços a um cliente de São Paulo atuando na automação de processos de extração de conhecimento em Diários Oficiais.

\section{Estrutura organizacional}

As atividades econômicas desenvolvidas pela empresa apresentam características que exigem a divisão de times em projetos, mesmo que alguns funcionários algumas vezes participem de mais de um projeto simultaneamente, como é o caso dos analistas de negócio por exemplo, a empresa se organiza para alocar os recursos temporariamente até sua liberação.

Atualmente a empresa trabalha com especialistas em áreas definidas, como desenvolvedores, administradores, designers, analistas de negócio e jornalistas. A maioria desses áreas é preenchida com apenas um profissional responsável por responder pelas demandas relacionadas ao seu domínio. Quando há a alocação de recursos para projetos, o responsável pela gerência do projeto analisa quais áreas serão relacionadas em que momento, e então é feita a alocação dos times.

A comunicação é feita de maneira informal devido a proximidade dos funcionários e ao pequeno quadro de colaboradores.

\section{Recursos disponíveis}

O time de desenvolvimento é composto por um desenvolvedor pleno e três estagiários incluindo o autor deste relatório. As áreas de design, análise de negócios e jornalismo contam, cada uma, com apenas 1 funcionário.

Devido a parceria que a empresa tem com a Universidade Federal de Ouro Preto, algumas vezes por ano ela recebe inter cambistas de experiência e área de atuação diversificada.

Dos recursos que são relevantes para o objeto de estudo deste trabalho, a empresa terceiriza o serviço de hospedagem, contando com dois servidores dedicados cuja especificações podem ser encontradas na \autoref{tab-spec-servidores}.

\begin{table}[htb]
	\caption{Especificações técnicas dos servidores da empresa}

	\label{tab-spec-servidores}	
\begin{tabular}{p{3.85cm}|p{5.20cm}|p{5.20cm}}
	%\hline
	 & \textbf{Servidor A} & \textbf{Servidor B}  \\
	\hline
	\textbf{CPU} & Intel(R) Core(TM) i5-4590 CPU @ 3.30GHz & AMD Opteron(tm) Processor 3280 \\
	\hline
	\textbf{Memória RAM} & 32 GB & 32 GB \\
	\hline
	\textbf{Disco rígido} & 500 GB & 500 GB \\
	% \hline
\end{tabular}
\end{table}

\section{O processo de desenvolvimento} 

As metodologias envolvidas nos projetos da empresa são inspirados em métodos ágeis, alguns utilizam SCRUM e outros apenas Kanban, isso varia de acordo com os requisitos e recursos disponíveis para cada projeto, mas de forma geral, a empresa sempre busca manter um quadro de atividades (Kanban) e realizar reuniões frequentes.

Através do acompanhamento do processo de desenvolvimento durante o período que este trabalho foi realizado, identifica-se um ciclo baseado em definição de requisitos, implementação e testes manuais. Inicialmente o coordenador do projeto realiza uma reunião com o time para especificar quais funcionalidades são prioritárias e detalhar o que precisa ser feito para seu desenvolvimento. Durante essa reunião é feita a divisão de atividades e a estipulação do tempo necessário por cada responsável. Uma pessoa é encarregada pelo registros das atividades no software de gestão, e em seguida cada membro do time começa a trabalhar na tarefa que lhe foi atribuída.

Diariamente o time realiza reuniões de acompanhamento, onde informam seu progresso ao coordenador do projeto e tiram dúvidas com os demais colaboradores. Caso uma atividade seja concluída, uma nova pode ser atribuída.

Durante o desenvolvimento, os programadores trabalham em \textit{branches} individuais com auxílio de um sistema de controle de versões. Eles testam manualmente cada funcionalidade em seu computador, e quando o time chega ao consentimento que existe um conjunto de funcionalidades pronto para ser entregue, algum membro do time compila o projeto em seu computador e realiza o envio do artefato gerado via SSH para o ambiente de produção. Durante essa etapa de entrega, caso a versão do artefato gerado apresente \textit{bugs}, o desenvolvedor responsável pela tarefa o corrige, gera uma nova versão e a envia novamente ao servidor de produção. Esta tarefa pode se repetir enquanto houverem \textit{bugs} que impossibilitem o correto funcionamento da aplicação.

Caso as características do artefato gerado não influenciem negativamente no processo dos usuários do sistema, um teste manual é realizado no ambiente de produção.

Quando surgem demandas imprevistas durante o ciclo de desenvolvimento, uma reunião é convocada para re-ajustarem o cronograma semanal e redistribuírem as tarefas. Caso um \textit{bugs} seja reportado pelo cliente, ele é avaliado quanto a sua criticidade e comparado à fila de tarefas pendentes. Caso seja um bug crítico, é tratado imediatamente, caso não seja critico e existam atividades prioritárias na fila, ele é registrado e levado em pauta na próxima reunião, onde o time reajusta o cronograma.

\subsection{Dados coletados com a análise}

A etapa de coleta de dados analíticos sobre o atual processo de desenvolvimento teve início no dia 8 de agosto de 2016. Durante essa etapa foram identificadas tarefas que fogem ao escopo de desenvolvimento e levantamento de requisitos, como atividades de compilação do código fonte, execução de testes manuais, configuração de ambientes de trabalho, etc. As execuções dessas atividades foram registradas a fim de sumarizar o tempo necessário e a quantidade de retrabalho envolvido. O acompanhamento e registro desses dados ocorreu até o dia 23 de setembro de 2016.

As seguintes tabelas apresentam dados quantitativos referentes as etapas do processo de desenvolvimento durante a fase de coleta. Graças a essas informações podemos identificar pontos de aprimoramento em potencial.

A \autoref{tab-code-analysis} apresenta dados analíticos referentes aos repositórios de código fonte da empresa.

\begin{table}[htb]
	\caption{Dados do repositório de código fonte no início do projeto}
	
	\label{tab-code-analysis}	
	\begin{tabular}{p{7.15cm}|p{7.10cm}}
		%\hline
		\textbf{Métrica} & \textbf{Valor}  \\
		\hline
		Número de \textit{commits} realizados & 125 \\
		\hline
		Número de artefatos gerados & 65 \\
		\hline
		Número de mesclagem de código realizados & 6 \\
		\hline
		Número de \textit{bugs} reportados pelo cliente & 20 \\
		\hline
		Número de atividades desenvolvidas pelo time que foram registradas no sistema de controle de atividades & 70 \\
		% \hline
	\end{tabular}
\end{table}


A \autoref{tab-deploys} demonstra o tempo necessário por entrega de uma nova versão do sistema. Consideramos o tempo necessário para construção de um artefato executável e a sua disponibilização em ambiente de produção. Esta etapa engloba tentativas má sucedidas de construção e entrega de uma versão estável e sem \textit{bugs}.

\begin{table}[htb]
	\caption{Características das entregas no início do projeto}
	
	\label{tab-deploys}	
	\begin{tabular}{p{3.85cm}|p{5.20cm}|p{5.20cm}}
		%\hline
		\textbf{Data da entrega} & \textbf{Tempo necessário} & \textbf{Número de tentativas má sucedidas até a conclusão da entrega}  \\
		\hline
		16/08/2016 & 2 horas & 0 \\
		\hline
		23/08/2016 & 6 horas & 3 \\
		\hline
		30/08/2016 & 3 horas & 1 \\
		\hline
		07/09/2016 & 6 horas & 5 \\
		\hline
		14/09/2016 & 8 horas & 7 \\
		\hline
		19/09/2016 & 4 horas & 2 \\
		% \hline
	\end{tabular}
\end{table}

Após a coleta de dados, uma entrevista foi realizada com o time de desenvolvimento onde foi solicitado que cada um citasse problemas marcantes que aconteceram com eles no decorrer do projeto com o objetivo de compreender a visão que o time tem sobre o processo de desenvolvimento em busca de identificar aspectos impactantes na rotina de trabalho que precisam ser melhorados. As seguintes respostas foram obtidas:

\begin{itemize}  
	\item Algum membro do time esqueceu de apagar um diretório que estava em lugar indevido, o que ocasionou a sobrecarga do disco-rígido do servidor do cliente, pois havia uma unidade de baixa quantidade de armazenamento disponível alocada apenas para o sistema operacional, e esta ficou inoperante, exigindo a investigação e remoção dos arquivos indevidos.
	\item Em uma tentativa de entrega de uma nova versão, foi identificado a existência de um diretório chamado "teste" no ambiente de produção. Consultando o time sobre a necessidade daquele diretório e na tentativa de apaga-lo do sistema pois não deveria estar ali, foi comunicado que o ambiente de produção estava neste diretório e não no lugar correto onde deveria ser executado.
	\item Em uma tentativa de entrega foi identificada a configuração indevida de permissões de um certo diretório feita por outro membro do time.
	\item Por diversas vezes o time identificou problemas relacionados à implantação de uma nova versão devido a arquivos de propriedades configurados incorretamente.
	\item Vários problemas já aconteceram devido a execução de \textit{scripts} no ambiente de produção sem os devidos testes serem realizados previamente.
	\item Diversos problemas aconteceram devido a necessidade da configuração manual do ambiente de produção, uma vez que diretórios precisavam ser manipulados por linha de comando.
\end{itemize}

\subsection{Características do processo atual} 

Baseado nos dados coletados durante a primeira fase do projeto identificamos como principais problemas o tempo necessário para entrega de uma nova versão estável do sistema e a inexistência de uma garantia de qualidade sobre os artefatos entregues. A qualidade neste contexto pode ser melhorada aumentando a capacidade do software atender a necessidade de mudanças, o que poderá ser identificado pela redução da quantidade de \textit{bugs}.

Segundo \citeonline{Flavia:AnaliseSistemas}, a qualidade de um software está diretamente relacionada à qualidade do processo de desenvolvimento, dessa forma, é comum que a busca por um software de maior qualidade passe necessariamente por uma melhoria no processo de desenvolvimento.

Relacionando as dificuldades reportadas pelo time com os problemas levantados, identificamos que as atividades que não são totalmente voltadas ao desenvolvimento nem ao levantamento de requisitos são os principais causadores dos problemas, provavelmente pela falta de especialistas em infraestrutura e pela falta de automação das atividades de implantação.

Os problemas citados se mostram relevantes uma vez que já foram identificados por especialistas, como \citeonline[p.~136]{Caelum:ArquiteturaJava} que descrevem um ambiente com características similares ao ambiente da empresa alvo de estudo deste trabalho como um ambiente problemático e sujeito a diversas falhas.

A partir de todos esses dados levantados, nossa linha investigativa foi elaborada em busca de reduzir o tempo necessário para implantação de novas versões do código, estabelecer métricas de qualidade, garantir que a versão de produção é estável e reduzir o número de versões defeituosas entregues pelo time.

% ---
% Capitulos de Procedimentos Experimentais
% ---
\chapter{Procedimentos Experimentais}

\section{Levantamento de metodologias e boas práticas}

\section{Adaptando a rotina do time para atender as demandas da empresa}

\section{Levantamento comparativo de ferramentas}

Para elaboração de um ambiente de desenvolvimento com tarefas de operações automatizadas, visando aumento da produtividade, economia de tempo e garantia de qualidade, foi realizado um levantamento comparativo entre ferramentas gratuitas ou de código fonte aberto que possibilitem a aplicação de todos os métodos necessários especificados pela nova metodologia de trabalho.

É importante que as ferramentas sejam integradas para agilizar a comunicação do time e que tenham interfaces de utilização que estejam de acordo com o novo processo de desenvolvimento estipulado.

Para resolver os problemas do cenário atual da empresa os seguintes tipos de ferramentas se fazem necessários:

\begin{itemize}
	\item Sistema de controle de versão com suporte a ramificações e pedidos de mesclagem de código submetidos a revisões preliminares.
	\item Sistema de execução de \textit{build pipelines} com suporte a ambientes de testes.
	\item Sistema de automação de implantações.
	\item Sistema de monitoramento de recursos.
\end{itemize}

Além dessas ferramentas citadas, para implantação de um ambiente de entrega contínua completo, tolerante a falhas e altamente escalável, um série de ferramentas adicionais se fazem necessárias, como centralizadores de logs, provisionadores de ambientes, orquestradores de máquinas virtuais, dentre outros. Devido a limitações de tempo, nem todas as ferramentas foram avaliadas e aplicadas no ambiente da empresa. Como o objetivo deste trabalho é resolver os principais problemas relacionados a produtividade e qualidade do processo de desenvolvimento, a maior quantidade de esforço foi concentrada na implantação das principais ferramentas e na adaptação da metodologia de trabalho do time.

Uma pesquisa foi realizada para selecionar as ferramentas disponíveis no mercado que atendem os requisitos levantados, pois como foi apontado pela \citeonline{XebiaLabs:DevOpsTools}, existem centenas de ferramentas criadas para estes propósitos, e testar todas tornaria este trabalho completamente inviável por limitações de tempo e pelo surgimento constante de novas ferramentas similares. Devido a estes fatores, para cada tipo de ferramenta foram levantadas entre 4 e 5 opções mais populares baseadas em análises de sites especialistas em cada assunto.

Diante do conjunto de ferramentas identificadas, as consideradas não cruciais para a boa conclusão deste trabalho não foram implantadas neste projeto. Por isso algumas das ferramentas descritas nos capítulos seguintes não apresentam um detalhamento tão aprofundado.

\subsection{Sistema de controle de versão}

Para escolher qual sistema de controle de versão usar levamos em conta que o time já possuía experiência prévia com \textit{git}, o que facilitaria sua utilização, e graças a análises feitas por \citeonline{DevMedia:VersionControl, InfoQ:VersionControl, SitePoint:VersionControl}, percebemos que o \textit{git} apresenta características que justificam ainda mais a sua escolha, mas ainda sim precisamos de uma ferramenta de controle de código fonte que tenha suporte a revisão e de preferência com uma interface gráfica para simplificar a gerência dos repositórios.

A \autoref{tab-git} mostra os resultados da pesquisa para as ferramentas mais populares de controle de versão baseadas em git. A escolha dessas ferramentas foi feita baseando-se em recomendações de outros profissionais e recorrendo a análises feitas por sites como \citeonline{Slant:SourceControl} e \citeonline{StackShare:SourceControl}.

\begin{table}[htb]
	\caption{Comparativo entre ferramentas de controle de versão de código baseadas em git disponíveis no mercado.}
	
	\label{tab-git}	
	\begin{tabular}{p{5.35cm}|p{2.0cm}|p{2.0cm}|p{2.0cm}|p{2.0cm}}
		%\hline
		 & \textbf{GitHub} & \textbf{BitBucket} & \textbf{Gogs} & \textbf{GitLab}  \\
		\hline
		\textbf{Hospedagem remota} & \checkmark & \checkmark & - & \checkmark \\
		\hline
		\textbf{Hospedagem local} & - & - & \checkmark & \checkmark \\
		\hline
		\textbf{Código aberto} & - & - & \checkmark & \checkmark \\
		\hline
		\textbf{Repositórios privados} & Pago & Limite de 5 usuários por repositório & \checkmark & \checkmark \\
		\hline
		\textbf{Gerência de organizações e times} & \checkmark & \checkmark & \checkmark & \checkmark \\
		\hline
		\textbf{Gerenciador de incidentes} & \checkmark & \checkmark & \checkmark & \checkmark \\
		\hline
		\textbf{Wiki} & \checkmark & \checkmark & \checkmark & \checkmark \\
		\hline
		\textbf{Pull requests} & \checkmark & \checkmark & Apenas entre um \textit{fork} e seu original & \checkmark \\
		\hline
		\textbf{Code Review} & \checkmark & \checkmark & Não permite comentários em trechos de código & \checkmark \\
		\hline
		\textbf{Web Hooks} & \checkmark & \checkmark & \checkmark & \checkmark \\
		\hline
		\textbf{Build pipeline} & - & \checkmark & - & \checkmark \\
		\hline
		\textbf{Pesquisa por trecho de código} & \checkmark & - & - & - \\
		\hline
		\textbf{Estatísticas de uso do repositório} & \checkmark & - & - & \checkmark \\
		\hline
		\textbf{Visualização gráfica das ramificações} & \checkmark & \checkmark & - & \checkmark \\
		\hline
		\textbf{Snippets} & \checkmark & \checkmark & - & \checkmark \\
		% \hline
	\end{tabular}
\end{table}

Baseando-se neste comparativo, chegamos a conclusão de que o GitHub e o BitBucket não representam a melhor escolha pelas limitações das versões gratuitas. Entre o Gogs e GitLab optamos pelo GitLab pela maior quantidade de recursos disponíveis.

\subsection{Ferramentas de execução de \textit{build pipelines}}

As ferramentas de execução de \textit{build pipelines} e de automação de rotinas de implantações podem ser encontradas na \autoref{tab-build-pipeline}. A seleção dessas ferramentas também foi feita levando-se em conta as análises feitas pelo \citeonline{StackShare:CI}, porém também foram inclusos o BitBucket e o GitLab devido ao fato de serem ferramentas que, além de darem suporte a estes requisitos, também realizam o controle de código fonte. Uma vez que é vantajoso ter menos ferramentas para administrar, elas tem preferência sobre as demais.

\begin{table}[htb]
	\caption{Comparativo entre ferramentas de execução de \textit{build pipelines} disponíveis no mercado.}
	
	\label{tab-build-pipeline}	
	\begin{tabular}{p{5.35cm}|p{2.0cm}|p{2.0cm}|p{2.0cm}|p{2.0cm}}
		%\hline
		& \textbf{Jenkins} & \textbf{GitLab} & \textbf{BitBucket} & \textbf{Travis CI}  \\
		\hline
		\textbf{Hospedagem remota} & - & \checkmark & \checkmark & \checkmark \\
		\hline
		\textbf{Hospedagem local} & \checkmark & \checkmark & - & - \\
		\hline
		\textbf{Código aberto} & \checkmark & \checkmark & - & - \\
		\hline
		\textbf{Construção de \textit{branches}} & \checkmark & \checkmark & \checkmark & \checkmark \\
		\hline
		\textbf{Construção de \textit{pull requestes}} & \checkmark & \checkmark & \checkmark & \checkmark \\
		\hline
		\textbf{Nível de complexidade} & Mediano & Simples & Simples & Simples \\
		\hline
		\textbf{Suporte a múltiplos ambientes} & \checkmark & \checkmark & Apenas variáveis de ambiente & Apenas variáveis de ambiente \\
		\hline
		\textbf{Integração com \textit{Docker} ou contêineres} & \checkmark & \checkmark & - & - \\
		\hline
		\textbf{Suporte a plugins} & \checkmark & - & - & - \\
		\hline
		\textbf{Suporte a rotinas de \textit{deploy}} & \checkmark & \checkmark & \checkmark & \checkmark \\
		\hline
		\textbf{Forma de definição da \textit{pipeline}} & Arquivo de configuração dentro do repositório ou através de uma interface gráfica & Arquivo de configuração dentro do repositório & Arquivo de configuração dentro do repositório & Arquivo de configuração dentro do repositório \\
		% \hline
	\end{tabular}
\end{table}

Após a análise das funcionalidades disponíveis, o BitBucket e o Travis CI foram descartados pelas limitações das versões gratuitas. Ao experimentarmos o Jenkins e o GitLab na prática, percebemos que a quantidade de \textit{plugins} disponíveis para o Jenkins fazem com que ele se torne extremamente flexível em relação a diversidade de projetos e permitem análises profundas de quesitos como cobertura de testes, documentação e relatórios de compilação. Por outro lado o GitLab se mostra simples e prático para projetos menores e com menor diversidade.

Devido a estes fatores optamos pela utilização das duas ferramentas, resolvemos executar a \textit{pipeline} no GitLab em projetos menores e menos complexos e deixamos o Jenkis para lidar com o processo de compilação de projetos maiores e mais complexos.

\subsection{Ferramentas de monitoramento}

As ferramentas de monitoramento de recursos, também identificadas com ajuda de análises de sites como \citeonline{DevOps:Monitoring} e \citeonline{StackShare:Monitoring} e de indicações de profissionais. Elas podem ser encontradas na \autoref{tab-monitoring}.

\begin{table}[htb]
	\caption{Comparativo entre ferramentas de monitoramento de recursos disponíveis no mercado.}
	
	\label{tab-monitoring}	
	\begin{tabular}{p{5.0cm}|p{1.6cm}|p{1.4cm}|p{2.4cm}|p{1.6cm}|p{1.4cm}}
		%\hline
		& \textbf{Zabbix} & \textbf{Cacti} & \textbf{Prometheus} & \textbf{Nagios} & \textbf{Icinga} \\
		\hline
		\textbf{Hospedagem remota} & - & - & - & - & - \\
		\hline
		\textbf{Hospedagem local} & \checkmark & \checkmark & \checkmark & \checkmark & \checkmark \\
		\hline
		\textbf{Código aberto} & \checkmark & \checkmark & \checkmark & \checkmark & \checkmark \\
		\hline
		\textbf{Geração de gráficos} & \checkmark & \checkmark & \checkmark & \checkmark & \checkmark \\
		\hline
		\textbf{Suporte a protocolo SNMP} & \checkmark & \checkmark &  \checkmark & \checkmark & \checkmark \\
		\hline
		\textbf{Suporte a protocolo JMX} & \checkmark &  \checkmark &  \checkmark & \checkmark & \checkmark \\
		\hline
		\textbf{Suporte a protocolo IPMI} & \checkmark & \checkmark &  \checkmark & \checkmark & \checkmark \\
		\hline
		\textbf{Cliente para coleta customizada} & \checkmark & \checkmark &  \checkmark & \checkmark & \checkmark \\
		\hline
		\textbf{Suporte a alertas e notificações de situações críticas} & \checkmark & - &  \checkmark & \checkmark & \checkmark \\
		% \hline
	\end{tabular}
\end{table}

Devido a pequena diferença em relação a quantidade de funcionalidades entre as ferramentas de monitoramento, utilizamos como critério a experiência do time com as ferramentas em busca de reduzir o tempo de aprendizado e optamos pelo Zabbix.

\subsection{Centralização de logs}

As ferramentas de centralização de logs foram deixadas de lado durante a execução deste projeto devido a limitações de tempo e pelo fato do time já realizar um bom trabalho administrando os logs gerados pela aplicação graças aos \textit{frameworks} de desenvolvimento utilizados. Porém uma breve análise de algumas das principais ferramentas de centralização de logs foi realizada para que um estudo de suas vantagens pudesse ser feito e essa decisão pudesse ser tomada.

Uma das ferramentas mais descomplicada e robusta para centralização de logs se chama Kibana \cite{Kibana:Logs}. O Kibana é uma aplicação \textit{web} com interface simples e amigável que permite a visualização de arquivos de log e estatísticas. Basicamente ele funciona como uma ferramenta de visualização em conjunto do Elastic Search, que é o responsável pelo armazenamento dos arquivos de log. Para utilização do Kibana é necessário configurar sua aplicação para enviar e armazenar os logs no Elastic Search, e com algumas configurações, preparar o Kiabana para receber consultas e exibir painéis informativos sobre as informações registradas.

Uma boa alternativa ao Kibana é o GrayLogs \cite{GrayLog:Logs}, que também funciona em conjunto do Elastic Search, porém uma das principais diferenças entre eles é que o GrayLog utiliza também o Apache Kafka como sistema intermediário de controle de mensagens que permite o particionamento do fluxo de dados entre \textit{clusters} de máquinas, o que é ideal para aplicações de analise de \textit{big data}. Diversas arquiteturas distribuídas podem se beneficiar desse modelo expandindo seu nível de controle e monitoramento do estado da aplicação.

\subsection{Conteinerização e provisionamento}

De acordo com os requisitos atuais dos projetos desenvolvidos pela empresa, apenas uma \textit{build pipeline} com \textit{deploy} automático configurado no Jenkins e no GitLab já se mostraram suficiente para atender as necessidades atuais, porem não podemos deixar de comentar sobre o uso de contêineres e provisionamento de máquinas virtuais, uma vez que essas tecnologias se mostram cada vez mais presentes em ambientes de desenvolvimento integrado, como pode ser observado em uma pesquisa publicada em 2016 pela \citeonline{RightScale:Docker}.

Os contêineres são resultado uma nova tecnologia de máquinas virtuais capazes de serem executadas sem a necessidade de carregamento de supervisores, o que as tornam mais leves que as máquinas virtuais convencionais. A utilização de contêineres para implantar um software em produção traz vantagens como controle total do ambiente de execução da aplicação, mais organização para configurações, versionamento de infra-estrutura, maior controle de dependências a nível de sistema operacional, maior aproveitamento do hardware, dentre outras \cite{JamesTurbull:DockerBook}.

Com a utilização de contêineres, o time de operações de uma empresa pode trabalhar com provisionamento de máquinas virtuais onde não há a necessidade de configuração manual de servidores, basta criar os \textit{scripts} de instanciação do contêiner, embutir a aplicação e inicia-lo no servidor de produção.

O provisionamento e implantação dos contêineres feito puramente com Docker muitas vezes não é tão simples quanto poderia ser, é aqui que entram as ferramentas de orquestração. As ferramentas de orquestração mais populares segundo a pesquisa da \citeonline{RightScale:Docker} são o Puppet e o Chef, que muitas das vezes são utilizadas em conjunto. Além dessas duas, outras bem populares são o 
Kubernetes \cite{Kubernetes:Orchestration} e o Open Shift \cite{OpenShift:Orchestration}.

Essas ferramentas permitem a construção de arquivos de descrição de infraestrutura, onde o desenvolvedor pode especificar quais imagens de contêineres serão instanciadas, quais serão os volumes de dados persistentes que ficarão salvos em disco, quais são as dependências entre contêineres para que o serviço funcione corretamente, quantas réplicas de cada serviço devem ser instanciadas, qual política de reinicialização deve ser aplicada a cada serviço em caso de falha, quais os requisitos de consumo para o levantamento de mais réplicas, dentre várias outras especificações.

Outra vantagem dessa abordagem é a construção de ambientes de testes e homologação idênticos ao ambiente de produção com uma quantidade mínima de trabalho necessário, reduzindo os riscos de comportamentos inesperados devido a ambientes diversificados.

\section{Implantação do ambiente de desenvolvimento integrado}



% ----------------------------------------------------------
% Capitulo com exemplos de comandos inseridos de arquivo externo 
% ----------------------------------------------------------

\include{abntex2-modelo-include-comandos}


% ----------------------------------------------------------
% Parte de revisãod e literatura
% ----------------------------------------------------------
\part{Resultados}

% ---
% Capitulo de revisão de literatura
% ---
\chapter{Análise do novo cenário}

\section{Coleta de dados}

\section{Comparativos com o cenário anterior}

\chapter{Conclusão}

\chapter{Sugestões para outras empresas}

\chapter{Trabalhos futuros}

\lipsum[1]

\lipsum[2-3]

% ---
% Finaliza a parte no bookmark do PDF
% para que se inicie o bookmark na raiz
% e adiciona espaço de parte no Sumário
% ---
\phantompart

% ---
% Conclusão
% ---
\chapter{Conclusão}
% ---

\lipsum[31-33]

% ----------------------------------------------------------
% ELEMENTOS PÓS-TEXTUAIS
% ----------------------------------------------------------
\postextual

% ----------------------------------------------------------
% Referências bibliográficas
% ----------------------------------------------------------
\bibliography{references}


% ----------------------------------------------------------
% Glossário
% ----------------------------------------------------------
%
% Consulte o manual da classe abntex2 para orientações sobre o glossário.
%
%\glossary

% ----------------------------------------------------------
% Apêndices
% ----------------------------------------------------------

% ---
% Inicia os apêndices
% ---
\begin{apendicesenv}

% Imprime uma página indicando o início dos apêndices
\partapendices

% ----------------------------------------------------------
\chapter{Quisque libero justo}
% ----------------------------------------------------------

\lipsum[50]

% ----------------------------------------------------------
\chapter{Nullam elementum urna vel imperdiet sodales elit ipsum pharetra ligula
ac pretium ante justo a nulla curabitur tristique arcu eu metus}
% ----------------------------------------------------------
\lipsum[55-57]

\end{apendicesenv}
% ---


% ----------------------------------------------------------
% Anexos
% ----------------------------------------------------------

% ---
% Inicia os anexos
% ---
\begin{anexosenv}

% Imprime uma página indicando o início dos anexos
\partanexos

% ---
\chapter{Morbi ultrices rutrum lorem.}
% ---
\lipsum[30]

% ---
\chapter{Cras non urna sed feugiat cum sociis natoque penatibus et magnis dis
parturient montes nascetur ridiculus mus}
% ---

\lipsum[31]

% ---
\chapter{Fusce facilisis lacinia dui}
% ---

\lipsum[32]

\end{anexosenv}

%---------------------------------------------------------------------
% INDICE REMISSIVO
%---------------------------------------------------------------------

\phantompart

\printindex

%---------------------------------------------------------------------
% Formulário de Identificação (opcional)
%---------------------------------------------------------------------
\chapter*[Formulário de Identificação]{Formulário de Identificação}
\addcontentsline{toc}{chapter}{Exemplo de Formulário de Identificação}
\label{formulado-identificacao}

Exemplo de Formulário de Identificação, compatível com o Anexo A (informativo)
da ABNT NBR 10719:2015. Este formulário não é um anexo. Conforme definido na
norma, ele é o último elemento pós-textual e opcional do relatório.

\bigskip

\begin{tabular}{|p{9cm}|p{5cm}|}
\hline
\multicolumn{2}{|c|}{\textbf{\large Dados do Relatório Técnico e/ou científico}}\\
\hline
\multirow{4}{10cm}[24pt]{Título e subtítulo}& Classificação de segurança\\
                   & \\
                   \cline{2-2}
                   & No.\\
                   & \\
				
\hline
Tipo de relatório & Data\\
\hline
Título do projeto/programa/plano & No.\\
\hline
\multicolumn{2}{|l|}{Autor(es)} \\
\hline
\multicolumn{2}{|l|}{Instituição executora e endereço completo} \\
\hline
\multicolumn{2}{|l|}{Instituição patrocinadora e endereço completo} \\
\hline
\multicolumn{2}{|l|}{Resumo}\\[3cm]
\hline
\multicolumn{2}{|l|}{Palavras-chave/descritores}\\
\hline
\multicolumn{2}{|l|}{
Edição \hfill No. de páginas \hfill No. do volume \hfill Nº de classificação \phantom{XXXX}} \\
\hline
\multicolumn{2}{|l|}{
ISSN \hfill \hfill Tiragem \hfill Preço \phantom{XXXXXXXX}} \\
\hline
\multicolumn{2}{|l|}{Distribuidor} \\
\hline
\multicolumn{2}{|l|}{Observações/notas}\\[3cm]
\hline
\end{tabular}

\end{document}