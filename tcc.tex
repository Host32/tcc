%% abtex2-modelo-relatorio-tecnico.tex, v<VERSION> laurocesar
%% Copyright 2012-<COPYRIGHT_YEAR> by abnTeX2 group at http://www.abntex.net.br/ 
%%
%% This work may be distributed and/or modified under the
%% conditions of the LaTeX Project Public License, either version 1.3
%% of this license or (at your option) any later version.
%% The latest version of this license is in
%%   http://www.latex-project.org/lppl.txt
%% and version 1.3 or later is part of all distributions of LaTeX
%% version 2005/12/01 or later.
%%
%% This work has the LPPL maintenance status `maintained'.
%% 
%% The Current Maintainer of this work is the abnTeX2 team, led
%% by Lauro César Araujo. Further information are available on 
%% http://www.abntex.net.br/
%%
%% This work consists of the files abntex2-modelo-relatorio-tecnico.tex,
%% abntex2-modelo-include-comandos and references.bib
%%

% ------------------------------------------------------------------------
% ------------------------------------------------------------------------
% abnTeX2: Modelo de Relatório Técnico/Acadêmico em conformidade com 
% ABNT NBR 10719:2015 Informação e documentação - Relatório técnico e/ou
% científico - Apresentação
% ------------------------------------------------------------------------ 
% ------------------------------------------------------------------------

\documentclass[
	% -- opções da classe memoir --
	12pt,				% tamanho da fonte
	openright,			% capítulos começam em pág ímpar (insere página vazia caso preciso)
	twoside,			% para impressão em recto e verso. Oposto a oneside
	a4paper,			% tamanho do papel. 
	% -- opções da classe abntex2 --
	%chapter=TITLE,		% títulos de capítulos convertidos em letras maiúsculas
	%section=TITLE,		% títulos de seções convertidos em letras maiúsculas
	%subsection=TITLE,	% títulos de subseções convertidos em letras maiúsculas
	%subsubsection=TITLE,% títulos de subsubseções convertidos em letras maiúsculas
	% -- opções do pacote babel --
	english,			% idioma adicional para hifenização
	french,				% idioma adicional para hifenização
	spanish,			% idioma adicional para hifenização
	brazil,				% o último idioma é o principal do documento
	]{abntex2}


% ---
% PACOTES
% ---

% ---
% Pacotes fundamentais 
% ---
\usepackage{lmodern}			% Usa a fonte Latin Modern
\usepackage[T1]{fontenc}		% Selecao de codigos de fonte.
\usepackage[utf8]{inputenc}		% Codificacao do documento (conversão automática dos acentos)
\usepackage{indentfirst}		% Indenta o primeiro parágrafo de cada seção.
\usepackage{color}				% Controle das cores
\usepackage{graphicx}			% Inclusão de gráficos
\usepackage{microtype} 			% para melhorias de justificação
% ---

% ---
% Pacotes adicionais, usados no anexo do modelo de folha de identificação
% ---
\usepackage{multicol}
\usepackage{multirow}
% ---
	
% ---
% Pacotes adicionais, usados apenas no âmbito do Modelo Canônico do abnteX2
% ---
\usepackage{lipsum}				% para geração de dummy text
% ---

% ---
% Pacotes de citações
% ---
\usepackage[brazilian,hyperpageref]{backref}	 % Paginas com as citações na bibl
\usepackage[alf]{abntex2cite}	% Citações padrão ABNT

% --- 
% CONFIGURAÇÕES DE PACOTES
% --- 

% ---
% Configurações do pacote backref
% Usado sem a opção hyperpageref de backref
\renewcommand{\backrefpagesname}{Citado na(s) página(s):~}
% Texto padrão antes do número das páginas
\renewcommand{\backref}{}
% Define os textos da citação
\renewcommand*{\backrefalt}[4]{
	\ifcase #1 %
		Nenhuma citação no texto.%
	\or
		Citado na página #2.%
	\else
		Citado #1 vezes nas páginas #2.%
	\fi}%
% ---

% ---
% Informações de dados para CAPA e FOLHA DE ROSTO
% ---
\titulo{Projeto de Automação de Operações e Entrega Contínua para Empresa de Desenvolvimento de Software.}
\autor{Ivan de Moura Miranda}
\local{Conselheiro Lafaiete}
\data{2016}
\instituicao{%
  Faculdade Presidente Antônio Carlos -- FUPAC
  \par
  Engenharia de Computação
  \par
  Prof. Me. Jean Carlo Mendes}
\tipotrabalho{Relatório técnico}
% O preambulo deve conter o tipo do trabalho, o objetivo, 
% o nome da instituição e a área de concentração 
\preambulo{Relatório técnico apresentado como trabalho de conclusão de curso para obtenção do diploma do curso de Engenharia de Computação, Faculdade Presidente Antônio Carlos de Conselheiro Lafaiete. }
% ---

% ---
% Configurações de aparência do PDF final

% alterando o aspecto da cor azul
\definecolor{blue}{RGB}{41,5,195}

% informações do PDF
\makeatletter
\hypersetup{
     	%pagebackref=true,
		pdftitle={\@title}, 
		pdfauthor={\@author},
    	pdfsubject={\imprimirpreambulo},
	    pdfcreator={LaTeX with abnTeX2},
		pdfkeywords={abnt}{latex}{abntex}{abntex2}{relatório técnico}, 
		colorlinks=true,       		% false: boxed links; true: colored links
    	linkcolor=blue,          	% color of internal links
    	citecolor=blue,        		% color of links to bibliography
    	filecolor=magenta,      		% color of file links
		urlcolor=blue,
		bookmarksdepth=4
}
\makeatother
% --- 

% --- 
% Espaçamentos entre linhas e parágrafos 
% --- 

% O tamanho do parágrafo é dado por:
\setlength{\parindent}{1.3cm}

% Controle do espaçamento entre um parágrafo e outro:
\setlength{\parskip}{0.2cm}  % tente também \onelineskip

% ---
% compila o indice
% ---
\makeindex
% ---

% ----
% Início do documento
% ----
\begin{document}

% Seleciona o idioma do documento (conforme pacotes do babel)
%\selectlanguage{english}
\selectlanguage{brazil}

% Retira espaço extra obsoleto entre as frases.
\frenchspacing 

% ----------------------------------------------------------
% ELEMENTOS PRÉ-TEXTUAIS
% ----------------------------------------------------------
% \pretextual

% ---
% Capa
% ---
\imprimircapa
% ---

% ---
% Folha de rosto
% (o * indica que haverá a ficha bibliográfica)
% ---
\imprimirfolhaderosto*
% ---


% ---
% RESUMO
% ---

% resumo na língua vernácula (obrigatório)
\setlength{\absparsep}{18pt} % ajusta o espaçamento dos parágrafos do resumo
\begin{resumo}
 Este trabalho descreve um projeto de otimização do processo de entrega de uma empresa de desenvolvimento de softwares. O objetivo é identificar as etapas de baixa produtividade para implementar ferramentas e aplicar metodologias que melhoram a qualidade dos softwares desenvolvidos. Durante o processo de desenvolvimento diversas tarefas repetitivas são executadas manualmente, com a automatização destas conseguimos reduzir o tempo investido no projeto, e com a aplicação de boas práticas de desenvolvimento, também garantimos melhor qualidade dos produtos desenvolvidos.

 \noindent
 \textbf{Palavras-chaves}: entrega contínua, integração contínua, \textit{devops}, metodologias ágeis.
\end{resumo}
% ---

% ---
% inserir lista de ilustrações
% ---
%\pdfbookmark[0]{\listfigurename}{lof}
%\listoffigures*
%\cleardoublepage
% ---

% ---
% inserir lista de tabelas
% ---
\pdfbookmark[0]{\listtablename}{lot}
\listoftables*
% ---

% ---
% inserir lista de abreviaturas e siglas
% ---
\begin{siglas}
	\item[AWS] \textit{Amazon Web Services}
	\item[BaaS] \textit{Back-end as a Service}
	\item[bug] \textit{...}
	\item[build] \textit{...}
	\item[devops] \textit{Development and Operations}
	\item[IaaS] \textit{Infrastructure as a Service}
	\item[PaaS] \textit{Platform as a Service}
	\item[SaaS] \textit{Software as a Service}
\end{siglas}
% ---

% ---
% inserir lista de símbolos
% ---
%\begin{simbolos}
%  \item[$ \Gamma $] Letra grega Gama
%  \item[$ \Lambda $] Lambda
%  \item[$ \zeta $] Letra grega minúscula zeta
%  \item[$ \in $] Pertence
%\end{simbolos}
% ---

% ---
% inserir o sumario
% ---
\pdfbookmark[0]{\contentsname}{toc}
\tableofcontents*
\cleardoublepage
% ---


% ----------------------------------------------------------
% ELEMENTOS TEXTUAIS
% ----------------------------------------------------------
\textual

\chapter{Introdução}
Segundo uma pesquisa publicada pelo \citeonline{StackOverflow:DeveloperSurvey2016}, dentre os maiores desafios encontrados por desenvolvedores em seu ambiente de trabalho vemos como mais influentes a documentação ruim, os requisitos mal especificados e o processo de desenvolvimento ineficiente.

Diante dessa realidade percebemos que o ambiente de desenvolvimento de softwares é negativamente afetado pela ineficiência do processo que muitas das vezes é excessivamente burocrático e pouco produtivo.

 \begin{citacao}
Mas ser burocrático não é o problema, pois as regras são necessárias para nortear qualquer tipo de trabalho. O problema reside, é claro, em ser excessivamente burocrático, gerando atividades/produtos que não agregam valor e consomem desnecessariamente tempo da equipe. \cite{EBusiness:ManifestoAgil}
\end{citacao}

Em busca de resolver estes e outros problemas, várias metodologias foram criadas, como os métodos ágeis, a cultura \textit{devops}, e as ferramentas que objetivam a automação dos processos de operação, permitindo a entrega contínua de software com a mínima intervenção humana, em busca de agilidade e qualidade do sistema entregue. O problema é que essa automação exige conhecimento de muitas tecnologias e ferramentas diferentes, o que é visto na maioria das vezes como barreira por desenvolvedores que não tem familiaridade com especificações de infraestrutura.

De outro lado, várias empresas viram uma oportunidade de negócio atrelada a resolução destes problemas, e diversas plataformas começaram a surgir nos últimos anos, como o \textit{Developer Cloud Service} \cite{Oracle:DevOps}, \textit{Open Shift} \cite{RedHat:OpenShift} e o \textit{AWS Lambda} \cite{Amazon:Lambda}. Estas soluções buscam facilitar a vida dos desenvolvedores oferecendo infraestrutura, plataforma ou \textit{back-end} como serviço para as empresas de desenvolvimento, e algumas ainda oferecem ferramentas para gestão dos projetos do time de desenvolvimento.

Apesar destas soluções prontas, o fato curioso é que, se não levarmos em conta o hardware e considerarmos apenas as ferramentas de automação fornecidas, boa parte delas são gratuitas e de código fonte aberto como o Git, Jenkins e Kubernetes. O próprio Open Shift por exemplo possui uma versão de código aberta que pode ser instalado na sua própria empresa com acesso a todos os recursos. Então a vantagem em contratar esse tipo de serviço se deve apenas a hospedagem na nuvem e ao baixo custo de manutenção dessas ferramentas que dispensam a necessidade de um profissional especializado contratado pela sua empresa, porém este custo ainda não é totalmente viável para micro e pequenas empresas, que de acordo com uma pesquisa realizada pela \citeonline{ABES:Pesquisa} representam 95,1\% das empresas de desenvolvimento e produção de tecnologia no Brasil.

Diante desse cenário, boa parte do mercado nacional pode se sentir desmotivado a contratar um serviço desse tipo, pois é muito difícil prever o benefício real proporcionado, afinal, o custo do investimento é fácil calcular, mas o lucro estimado é complicado, uma vez que eficiência e eficácia precisam ser convertidas em dinheiro baseadas na confiança de que a automação irá realmente resolver os problemas da empresa.

Uma alternativa a este investimento ainda existe, que é a capacitação profissional do time de desenvolvimento para a utilização das ferramentas gratuitas em servidores locais, que não precisam ser computadores muito potentes devido a baixa necessidade de escalabilidade e performance, uma vez que os usuários desses sistemas serão os próprios colaboradores da empresa.

Para implantação desse ambiente automatizado que utiliza apenas ferramentas gratuitas, é necessário o estudo de uma gama de soluções a fim de identificar a que mais atende a real necessidade da empresa.

% ---
% Capitulos de Objetivo Geral
% ---
\section{Objetivos}

Em uma tentativa de validar os benefícios da automação e entrega contínua num cenário real, graças ao apoio da DevelOP, uma empresa de desenvolvimento de software situada em Ouro Preto, este trabalho se propõe a implementar um ambiente de desenvolvimento integrado baseado no estudo de boas práticas e ferramentas gratuitas existentes no mercado buscando a automação das tarefas de operações e o aumento da qualidade dos softwares desenvolvidos. Objetiva-se também avaliar a eficácia dos métodos propostos graças a uma analise da performance da empresa antes e depois da implementação do projeto.

% ---
% Capitulos de Objetivos Específicos
% ---
\section{Organização dos capítulos}

% ----------------------------------------------------------
% PARTE - preparação da pesquisa
% ----------------------------------------------------------
\part{Preparação do relatório}

% ---
% Capitulos de Metodologia
% ---
\chapter{Metodologia}

A fim de avaliar os benefícios das soluções propostas, este trabalho será fundamentado na investigação a respeito do tema. As situações referentes ao objeto de estudo, que no caso se trata do processo de desenvolvimento de softwares da empresa DevelOP, serão examinadas com olhar investigativo buscando atingir a maior veracidade possível.

O projeto visa abordar o conhecimento a respeito da gerência de projetos, desenvolvimento e entrega de softwares e para isso se faz necessário direcionar a abordagem com base na utilização de material teórico, estabelecendo uma linha de investigação pela qual será conduzido o trabalho para que sejam levantado todo o material necessário com o intuito de estabelecer uma avaliação prática dos resultados obtidos.

Após o levantamento do material teórico o projeto será divido em 3 etapas. A primeira se baseia no estudo do cenário atual da empresa coletando e analisando dados em busca de identificar pontos problemáticos e levantar propostas de soluções. A segunda fase terá por objetivo implementar as soluções propostas e analisar seus efeitos no ambiente da empresa. A terceira fase terá a responsabilidade de mostrar os resultados e avaliar as soluções oferecidas e as vantagens que propõe.

Este trabalho visa atuar utilizando o método experimental de forma a estudar os fatores que influenciam o tempo e a qualidade do processo de desenvolvimento, desta forma avaliando a veracidade das hipóteses levantadas.

\section{Procedimentos para coleta e análise dos dados}

A coleta dos dados será feita através do acompanhamento das rotinas da empresa para que o conhecimento da área no qual está focado o tema seja ampliado e permitindo uma análise mais enfática quando se necessita recolher informações de variados aspectos relacionados, assim preenchendo melhor as lacunas no projeto.

A análise dos dados será feita atrás do comparativo entre o antes e depois da implementação das soluções propostas. Comparando o tempo necessário para entregar uma nova versão do software, a relação entre a quantidade \textit{builds} quebradas e bem sucedidas por dia, a quantidade de  \textit{bugs} reportados por semana, a facilidade de manutenção do código e a satisfação dos desenvolvedores com o processo de desenvolvimento.


% ---
% Capitulos de desenvolvimento
% ---
\chapter{Análise do cenário atual}

\section{Associação de Desenvolvimento de Software, Produtos e Pessoas da Região dos Inconfidentes Mineiros - DevelOP}

A DevelOP é uma associação privada fundada em 2016 por um grupo de profissionais em tecnologia da informação que desenvolve softwares, produtos e pessoas na região dos inconfidentes mineiros. (Pegar mais informações sobre a fundação com o Álvaro)

Situada no centro de Ouro Preto, a empresa tem como principal atividade econômica o desenvolvimento de programas de computador sob encomenda, mas também atua em atividades de desenvolvimento e licenciamento de programas de computador customizável, consultoria em tecnologia da informação e atividades de apoio à educação.

(Visão e Missão.)

Apesar de ser uma empresa nova no mercado, a DevelOP já possui alguns projetos em seu portfólio, como o evento Empreenda. Em Ação! realizado em julho de 2016 onde equipes formadas por alunos de disciplinas de Empreendedorismo da Universidade Federal de Ouro Preto (UFOP) competem na idealização, planejamento e apresentação de modelos de negócio reais e o atual projeto em desenvolvimento, onde a empresa presta serviços a um cliente de São Paulo atuando na automação de processos de extração de conhecimento em Diários Oficiais.

\section{Estrutura organizacional}

As atividades econômicas desenvolvidas pela empresa apresentam características que exigem a divisão de times em projetos, mesmo que alguns funcionários algumas vezes participem de mais de um projeto simultaneamente, como é o caso dos analistas de negócio por exemplo, a empresa se organiza para alocar os recursos temporariamente até sua liberação.

Atualmente a empresa trabalha com especialistas em áreas definidas, como desenvolvedores, administradores, designers, analistas de negócio e jornalistas. A maioria desses áreas é preenchida com apenas um profissional responsável por responder pelas demandas relacionadas ao seu domínio. Quando há a alocação de recursos para projetos, o responsável pela gerência do projeto analisa quais áreas serão relacionadas em que momento, e então é feita a alocação dos times.

A comunicação é feita de maneira informal devido a proximidade dos funcionários e ao pequeno quadro de colaboradores.

\section{Recursos disponíveis}

O time de desenvolvimento é composto por (cargo do ivan), (cargo do henrique) e dois estagiários. As áreas de design, análise de negócios e jornalismo contam, cada uma, com apenas 1 funcionário.

Devido a parceria que a empresa tem com a Universidade Federal de Ouro Preto, algumas vezes no ano ela recebe a presença de inter cambistas de experiência e área de atuação diversificada.

Dos recursos que são relevantes para o objeto de estudo deste trabalho, a empresa terceiriza o serviço de hospedagem, contando com dois servidores dedicados de especificações idênticas que podem ser encontradas na \autoref{tab-spec-servidores}.

\begin{table}[htb]
	\caption{Especificações técnicas dos servidores da empresa}

	\label{tab-spec-servidores}	
\begin{tabular}{p{3.85cm}|p{5.20cm}|p{5.20cm}}
	%\hline
	 & \textbf{Servidor A} & \textbf{Servidor B}  \\
	\hline
	\textbf{CPU} & Intel(R) Core(TM) i5-4590 CPU @ 3.30GHz & Intel(R) Core(TM) i5-4590 CPU @ 3.30GHz \\
	\hline
	\textbf{Memória RAM} & 32 GB & 32 GB \\
	\hline
	\textbf{Disco rígido} & 500 GB & 500 GB \\
	% \hline
\end{tabular}
\end{table}

\section{O processo de desenvolvimento} 

As metodologias envolvidas nos projetos da empresa são inspirados em métodos ágeis, alguns utilizam SCRUM e outros apenas Kanban, isso varia de acordo com os requisitos e recursos disponíveis para cada projeto, mas de forma geral, a empresa sempre busca manter um quadro de atividades (Kanban) e realizar reuniões frequentes.



% ---
% Capitulos de Procedimentos Experimentais
% ---
\chapter{Procedimentos Experimentais}

\section{Estudo de metodologias e boas práticas de gestão}

\section{Adaptando a rotina do time para atender as demandas da empresa}

\section{Levantamento e comparativo de ferramentas}

\section{Implantação do ambiente de desenvolvimento integrado}

% ----------------------------------------------------------
% Capitulo com exemplos de comandos inseridos de arquivo externo 
% ----------------------------------------------------------

\include{abntex2-modelo-include-comandos}


% ----------------------------------------------------------
% Parte de revisãod e literatura
% ----------------------------------------------------------
\part{Resultados}

% ---
% Capitulo de revisão de literatura
% ---
\chapter{Análise do novo cenário}

\section{Coleta de dados}

\section{Comparativos com o cenário anterior}

\chapter{Conclusão}

\chapter{Sugestões para outras empresas}

\chapter{Trabalhos futuros}

\lipsum[1]

\lipsum[2-3]

% ---
% Finaliza a parte no bookmark do PDF
% para que se inicie o bookmark na raiz
% e adiciona espaço de parte no Sumário
% ---
\phantompart

% ---
% Conclusão
% ---
\chapter{Conclusão}
% ---

\lipsum[31-33]

% ----------------------------------------------------------
% ELEMENTOS PÓS-TEXTUAIS
% ----------------------------------------------------------
\postextual

% ----------------------------------------------------------
% Referências bibliográficas
% ----------------------------------------------------------
\bibliography{references}


% ----------------------------------------------------------
% Glossário
% ----------------------------------------------------------
%
% Consulte o manual da classe abntex2 para orientações sobre o glossário.
%
%\glossary

% ----------------------------------------------------------
% Apêndices
% ----------------------------------------------------------

% ---
% Inicia os apêndices
% ---
\begin{apendicesenv}

% Imprime uma página indicando o início dos apêndices
\partapendices

% ----------------------------------------------------------
\chapter{Quisque libero justo}
% ----------------------------------------------------------

\lipsum[50]

% ----------------------------------------------------------
\chapter{Nullam elementum urna vel imperdiet sodales elit ipsum pharetra ligula
ac pretium ante justo a nulla curabitur tristique arcu eu metus}
% ----------------------------------------------------------
\lipsum[55-57]

\end{apendicesenv}
% ---


% ----------------------------------------------------------
% Anexos
% ----------------------------------------------------------

% ---
% Inicia os anexos
% ---
\begin{anexosenv}

% Imprime uma página indicando o início dos anexos
\partanexos

% ---
\chapter{Morbi ultrices rutrum lorem.}
% ---
\lipsum[30]

% ---
\chapter{Cras non urna sed feugiat cum sociis natoque penatibus et magnis dis
parturient montes nascetur ridiculus mus}
% ---

\lipsum[31]

% ---
\chapter{Fusce facilisis lacinia dui}
% ---

\lipsum[32]

\end{anexosenv}

%---------------------------------------------------------------------
% INDICE REMISSIVO
%---------------------------------------------------------------------

\phantompart

\printindex

%---------------------------------------------------------------------
% Formulário de Identificação (opcional)
%---------------------------------------------------------------------
\chapter*[Formulário de Identificação]{Formulário de Identificação}
\addcontentsline{toc}{chapter}{Exemplo de Formulário de Identificação}
\label{formulado-identificacao}

Exemplo de Formulário de Identificação, compatível com o Anexo A (informativo)
da ABNT NBR 10719:2015. Este formulário não é um anexo. Conforme definido na
norma, ele é o último elemento pós-textual e opcional do relatório.

\bigskip

\begin{tabular}{|p{9cm}|p{5cm}|}
\hline
\multicolumn{2}{|c|}{\textbf{\large Dados do Relatório Técnico e/ou científico}}\\
\hline
\multirow{4}{10cm}[24pt]{Título e subtítulo}& Classificação de segurança\\
                   & \\
                   \cline{2-2}
                   & No.\\
                   & \\
				
\hline
Tipo de relatório & Data\\
\hline
Título do projeto/programa/plano & No.\\
\hline
\multicolumn{2}{|l|}{Autor(es)} \\
\hline
\multicolumn{2}{|l|}{Instituição executora e endereço completo} \\
\hline
\multicolumn{2}{|l|}{Instituição patrocinadora e endereço completo} \\
\hline
\multicolumn{2}{|l|}{Resumo}\\[3cm]
\hline
\multicolumn{2}{|l|}{Palavras-chave/descritores}\\
\hline
\multicolumn{2}{|l|}{
Edição \hfill No. de páginas \hfill No. do volume \hfill Nº de classificação \phantom{XXXX}} \\
\hline
\multicolumn{2}{|l|}{
ISSN \hfill \hfill Tiragem \hfill Preço \phantom{XXXXXXXX}} \\
\hline
\multicolumn{2}{|l|}{Distribuidor} \\
\hline
\multicolumn{2}{|l|}{Observações/notas}\\[3cm]
\hline
\end{tabular}

\end{document}