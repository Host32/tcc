%% abtex2-modelo-slides.tex, v-1.0 gfabinhomat
%% Copyright 2012-<COPYRIGHT_YEAR> by abnTeX2 group at http://www.abntex.net.br/ 
%%
%% This work may be distributed and/or modified under the
%% conditions of the LaTeX Project Public License, either version 1.3
%% of this license or (at your option) any later version.
%% The latest version of this license is in
%%   http://www.latex-project.org/lppl.txt
%% and version 1.3 or later is part of all distributions of LaTeX
%% version 2005/12/01 or later.
%%
%% This work has the LPPL maintenance status `maintained'.
%% 
%% The Current Maintainer of this work is Fábio Rodrigues Silva, 
%% member of abnTeX2 team, led by Lauro César Araujo. 
%% Further information are available on 
%% http://www.abntex.net.br/
%%
%% This work consists of the files abntex2-modelo-slides.tex, 
%% abntex2-modelo-references.bib and abntex2-modelo-marca.pdf
%%
%% Modelo desenvolvido por Fábio Rodrigues Silva (gfabinhomat@gmail.com)
%% Mais informações podem ser obtidas no guia do usuário Beamer 
%% (http://linorg.usp.br/CTAN/macros/latex/contrib/beamer/doc/beameruserguide.pdf)
%% Informações rápidas podem ser acessadas em http://en.wikibooks.org/wiki/LaTeX/Presentations


% Apresentações em widescreen. Outros valores possíveis: 1610, 149, 54, 43 e 32.
% Por padrão, as apresentações são no formato 4:3 (sem o aspectratio).
\documentclass[aspectratio=169]{beamer}	 	

\usetheme{Pittsburgh}
\usecolortheme{default}
\usefonttheme[onlymath]{serif}			% para fontes matemáticas
% Enconte mais temas e cores em http://www.hartwork.org/beamer-theme-matrix/ 
% Veja também http://deic.uab.es/~iblanes/beamer_gallery/index.html

% Customizações de Cores: fg significa cor do texto e bg é cor do fundo
\setbeamercolor{normal text}{fg=black}
\setbeamercolor{alerted text}{fg=red}
\setbeamercolor{author}{fg=blue}
\setbeamercolor{institute}{fg=blue}
\setbeamercolor{date}{fg=green}
\setbeamercolor{frametitle}{fg=red}
\setbeamercolor{framesubtitle}{fg=brown}
\setbeamercolor{block title}{bg=blue, fg=white}		%Cor do título
\setbeamercolor{block body}{bg=gray, fg=darkgray}	%Cor do texto (bg= fundo; fg=texto)

\setbeamertemplate{caption}[numbered]

% ---
% PACOTES
% ---
\usepackage[alf]{abntex2cite}		% Citações padrão ABNT
\usepackage[brazil]{babel}		% Idioma do documento
\usepackage{color}			% Controle das cores
\usepackage[T1]{fontenc}		% Selecao de codigos de fonte.
\usepackage{graphicx}			% Inclusão de gráficos
\usepackage[utf8]{inputenc}		% Codificacao do documento (conversão automática dos acentos)
\usepackage{txfonts}			% Fontes virtuais
% ---

% --- Informações do documento ---
\title{Projeto de Automação de Operações e Entrega Contínua para Empresa de Desenvolvimento de Software}
\author{Ivan de Moura Miranda}
\institute{Faculdade Presidente Antônio Carlos
	    \par
	    Engenharia de Computação}
\date{\today, v1}
% ---

% ----------------- INÍCIO DO DOCUMENTO --------------------------------------
\begin{document}

% ----------------- NOVO SLIDE --------------------------------
\begin{frame}

\begin{minipage}{1\linewidth}
  \centering
  \begin{tabular}{cc}
    \begin{tabular}{c}
    \end{tabular}
    &
    \begin{tabular}{c}
      \textbf{Faculdade Presidente Antônio Carlos} \\ \textbf{Engenharia de Computação}
    \end{tabular}
  \end{tabular}
\end{minipage}

\titlepage

\end{frame}

% ----------------- NOVO SLIDE --------------------------------
\begin{frame}{Sumário}
\tableofcontents
\end{frame}

% ----------------- NOVO SLIDE --------------------------------
\section{Introdução}

\begin{frame}{Introdução}

Maiores desafios encontrados por desenvolvedores em seu ambiente de trabalho \cite{StackOverflow:DeveloperSurvey2016}:

\begin{itemize}
	\item Expectativas irrealistas
	
	\item Documentação ruim
	
	\item Requisitos mal especificados
	
	\item Processo de desenvolvimento ineficiente
\end{itemize}

\end{frame}

\begin{frame}{Introdução}
	
	Soluções existentes:
	
	\begin{itemize}
		
		\item Modelos e \textit{frameworks} de boas práticas:
		
		\begin{itemize}
			\item \textit{ITIL}
			\item \textit{COBIT} 
			\item \textit{CMMI} 
		\end{itemize}
		
		\item Metodologias ágeis:
		
		\begin{itemize}
			\item \textit{SCRUM}
			\item \textit{XP}
			\item \textit{DevOps}
		\end{itemize}
		
	\end{itemize}
	
\end{frame}

\begin{frame}{Introdução}
	
	\framesubtitle{Objetivos}
	
	\begin{itemize}
		
		\item Modificar o ambiente de desenvolvimento de uma empresa baseando-se no estudo de boas práticas e ferramentas gratuitas buscando a automação das tarefas de operações e o aumento da qualidade dos softwares desenvolvidos
		
		\item Validar os benefícios da automação e entrega contínua em um cenário real
		
		\item Avaliar os métodos propostos
		
	\end{itemize}
	
\end{frame}


\section{Situação da empresa no início do projeto}

\begin{frame}
	\frametitle{DevelOP}
	\framesubtitle{Associação de Desenvolvimento de \textit{Software}, Produtos e Pessoas da Região dos Inconfidentes Mineiros}
	
	\begin{itemize}
		
		 \item Associação de direito privado, constituída na forma de sociedade civil de fins não lucrativos, com autonomia administrativa e financeira, fundada em 2016
		 
		 \item Tem por finalidade central realizar ações sociais de utilidade pública na área de desenvolvimento de \textit{software}, produtos de \textit{software} e pessoas \cite{DevelOP:Estatuto}
		 
		 \item Situada no centro de Ouro Preto

	\end{itemize}

\end{frame}

% ----------------- NOVO SLIDE --------------------------------

\begin{frame}
	\frametitle{DevelOP}
	\framesubtitle{Estrutura organizacional}
	
	\begin{itemize}
		
		\item Divisão de equipes por projeto
		
		\item Comunicação informal
		
		\item Setores pequenos formados por especialistas
		
	\end{itemize}
	
\end{frame}

% ----------------- NOVO SLIDE --------------------------------

\begin{frame}
	
	\frametitle{Situação da empresa no início do projeto}
	\framesubtitle{O processo de desenvolvimento}
	
	\begin{itemize}
		
		\item Aplicadas algumas práticas do \textit{SCRUM} como \textit{product backlog}, divisão do projeto em \textit{Sprints} e reuniões diárias
		
		\item Ciclo baseado em definição de requisitos, implementação e testes manuais
		
		\item Compilação de código fonte no computador do desenvolvedor
		
		\item Processo de implantação de novas versões totalmente manual
		
		\item Espera acumular muitas alterações para realizar a entrega
	\end{itemize}
	
\end{frame}

% ----------------- NOVO SLIDE --------------------------------

\begin{frame}
	
	\frametitle{Situação da empresa no início do projeto}
	\framesubtitle{Dados coletados}
	
	\begin{table}[htb]
		\caption{Dados do repositório de código fonte no início do projeto}
		
		\label{tab-code-analysis}	
		\begin{tabular}{p{7.15cm}|p{6.0cm}}
			%\hline
			\textbf{Métrica} & \textbf{Quantidade}  \\
			\hline
			Número de \textit{commits} realizados & 125 \\
			\hline
			Número de artefatos gerados & 65 \\
			\hline
			Número de mesclagem de código realizados & 6 \\
			\hline
			Número de \textit{bugs} reportados pelo cliente & 20 \\
			\hline
			Número de atividades desenvolvidas pelo time que foram registradas no sistema de controle de atividades & 70 \\
			% \hline
		\end{tabular}
	\end{table}
	
\end{frame}

\begin{frame}
	
	\frametitle{Situação da empresa no início do projeto}
	\framesubtitle{Dados coletados}
	
	\begin{table}[htb]
		\caption{Características das entregas no início do projeto}
		
		\label{tab-deploys}	
		\begin{tabular}{p{3.5cm}|p{3.5cm}|p{5.50cm}}
			%\hline
			\textbf{Data da entrega} & \textbf{Tempo necessário} & \textbf{Número de tentativas má sucedidas até a conclusão da entrega}  \\
			\hline
			16/08/2016 & 2 horas & 0 \\
			\hline
			23/08/2016 & 6 horas & 3 \\
			\hline
			30/08/2016 & 3 horas & 1 \\
			\hline
			07/09/2016 & 6 horas & 5 \\
			\hline
			14/09/2016 & 8 horas & 7 \\
			\hline
			19/09/2016 & 4 horas & 2 \\
			% \hline
		\end{tabular}
	\end{table}
	
\end{frame}


\begin{frame}
	
	\frametitle{Situação da empresa no início do projeto}
	\framesubtitle{Problemas do cotidiano}
	
	\begin{itemize}
		
		\item Esquecer de apagar diretórios de locais indevidos
		
		\item Executar a versão de produção em diretórios incorretos
		
		\item Permissões indevidas
		
		\item Dificuldade em gerenciar configurações
		
		\item Execução de \textit{scripts} em produção sem os devidos testes anteriormente
	\end{itemize}
	
\end{frame}

% ----------------- NOVO SLIDE --------------------------------
\section{Trabalho realizado}

\begin{frame}
	\frametitle{Trabalho realizado}
	\framesubtitle{Automação das tarefas de entrega}
	
	\begin{itemize}
		 \item O \textit{updload} de novas versões de forma manual é comprometido pela velocidade e estabilidade da conexão de internet no escritório

		 \item Decidido compilar o código em um servidor remoto, sem problemas relacionados a conexão de internet
		 
		 \item Necessário garantir a estabilidade da base de código antes de automatizar este processo
	\end{itemize}

\end{frame}

\begin{frame}
	\frametitle{Trabalho realizado}
	\framesubtitle{Garatindo estabilidade a base de código}
	
	\begin{itemize}
		
		\item Testes automatizados
		
		\item Refatoração constante
		
		\item \textit{Code review}
	
		\item Integração contínua
		
	\end{itemize}
	
\end{frame}

\begin{frame}
	\frametitle{Trabalho realizado}
	\framesubtitle{Entrega contínua}
	
	\begin{itemize}
		
		\begin{columns}
			\begin{column}{0.5\textwidth}
				\item Entrega de \textit{software} antes:
				
					\begin{itemize}
						
						\item Acumulava várias atualizações para realizar entregas com baixa frequência.
						
						\item Picos de esforço relacionados a entrega
						
						\item Muitos alterações para investigar a causa de \textit{bugs}
						
						\item Alto risco
						
					\end{itemize}
			\end{column}
			\begin{column}{0.5\textwidth} 
			
				\item Entrega de software depois:
				
					\begin{itemize}
						
						\item Consegue entregar pequenas atualizações com alta frequência
						
						\item Pouco esforço relacionado a cada entrega
						
						\item Poucas alterações para investigar a causa de \textit{bugs}
						
						\item Baixo risco
						
					\end{itemize}
				\end{column}
			\end{columns}
		
	\end{itemize}
	
\end{frame}

\begin{frame}
	\frametitle{Trabalho realizado}
	\framesubtitle{Ferramentas utilizadas}
	
	\begin{itemize}
		
		\item Docker 
		
		\item Gitlab
		
		\item Jenkins
		
		\item Zabbix
		
	\end{itemize}
	
\end{frame}

\section{Resultados obtidos}

\begin{frame}
	\frametitle{Resultados}
	\framesubtitle{Dados coletados}
	
	\begin{table}[htb]
		\caption{Dados do repositório de código fonte no fim do projeto}
		
		\label{tab-code-analysis-new}	
		\begin{tabular}{p{7.15cm}|p{5.10cm}}
			%\hline
			\textbf{Métrica} & \textbf{Quantidade}  \\
			\hline
			Número de \textit{commits} realizados & 434 \\
			\hline
			Número de artefatos gerados & 115 \\
			\hline
			Número de mesclagem de código realizados & 19 \\
			\hline
			Número de mesclagem de código rejeitadas & 3 \\
			\hline
			Número de \textit{bugs} reportados pelo cliente & 6 \\
			\hline
			Número de atividades desenvolvidas pelo time que foram registradas no sistema de controle de atividades & 34 \\
			\hline
			Tempo médio de execução da \textit{build pipeline} & 5 minutos \\
			% \hline
		\end{tabular}
	\end{table}
	
\end{frame}


\begin{frame}
	\frametitle{Resultados}
	\framesubtitle{Dados coletados}
	
	\begin{table}[htb]
		\caption{Características das entregas no fim do projeto}
		
		\label{tab-deploys-new}	
		\begin{tabular}{p{3.0cm}|p{5.0cm}|p{5.20cm}}
			%\hline
			\textbf{Data da entrega} & \textbf{Tempo necessário} & \textbf{Número de tentativas má sucedidas até a conclusão da entrega}  \\
			\hline
			14/10/2016 & 13 segundos & 0 \\
			\hline
			18/10/2016 & 17 segundos & 0 \\
			\hline
			23/10/2016 & 1 minuto e 21 segundos & 0 \\
			\hline
			25/10/2016 & 1 minuto e 17 segundos & 0 \\
			\hline
			26/10/2016 & 16 segundos & 0 \\
			\hline
			28/10/2016 & 12 segundos & 0 \\
			\hline
			31/10/2016 & 14 segundos & 0 \\
			% \hline
		\end{tabular}
	\end{table}
	
\end{frame}

\begin{frame}
	\frametitle{Resultados}
	\framesubtitle{Comparativo}
	
	\begin{table}[htb]
		\caption{Comparativo baseado em uma média mensal dos dados coletados antes e depois da conclusão do projeto.}
		
		\label{tab-comparativo}	
		\begin{tabular}{p{7.85cm}|p{2.5cm}|p{2.5cm}}
			%\hline
			& \textbf{Antes} & \textbf{Depois}  \\
			\hline
			\textbf{Tempo médio necessário para realizar a entrega de uma nova versão} & 5 horas & 33 segundos \\
			\hline
			\textbf{Média de tentativas má sucedidas até a conclusão de uma entrega} & 3 & 0 \\
			\hline
			\textbf{Número de commits realizados} & 62 & 434 \\
			\hline
			\textbf{Número de artefatos gerados} & 32 & 115 \\
			\hline
			\textbf{Número de mesclagem de código realizadas} & 3 & 19 \\
			\hline
			\textbf{Número de mesclagem de código rejeitadas} & 1 & 3 \\
			\hline
			\textbf{Número de bugs reportados pelo cliente} & 10 & 6 \\
			% \hline
		\end{tabular}
	\end{table}
	
\end{frame}


\begin{frame}
	\frametitle{Resultados}
	\framesubtitle{\textit{Feedback} da equipe de desenvolvimento}
	
	\begin{itemize}
		
		\item Realizando alterações pequenas e constantes é mais fácil identificar e corrigir \textit{bugs}
		
		\item Não precisa dedicar um dia para fazer a entrega pois é totalmente automatizada
		
		\item Maior satisfação do cliente com a redução do tempo de resposta para correção de \textit{bugs} 
		
		\item O \textit{code review} se mostra burocrático em situações de urgência, mas colabora com o compartilhamento do conhecimento sobre a base de códigos.
		
	\end{itemize}
	
\end{frame}

\begin{frame}
	\frametitle{Trabalhos futuros}
	
	\begin{itemize}
		
		\item Centralização de \textit{logs}
		
		\item Uso de orquestradores para contêineres
		
		\item Implantação contínua
		
		\item Provisionamento de \textit{hardware}
		
	\end{itemize}
	
\end{frame}

\begin{frame}
	\frametitle{Conclusão}
	
	\begin{itemize}
		
		\item Metodologias ágeis são extremamente vantajosas quando aplicadas corretamente
		
		\item A automação de tarefas manuais economiza muito tempo investido no projeto
		
		\item Existem muitas ferramentas gratuitas eficientes para alcançar os objetivos desejados
		
		\item O principal fator para o estabelecimento de um ambiente de desenvolvimento eficaz é a colaboração de todos os membros do time e a busca constante por automação e melhoria do processo
		
	\end{itemize}
	
\end{frame}

% ----------------- NOVO SLIDE --------------------------------

% --- O comando \allowframebreaks ---
% Se o conteúdo não se encaixa em um quadro, a opção allowframebreaks instrui 
% beamer para quebrá-lo automaticamente entre dois ou mais quadros,
% mantendo o frametitle do primeiro quadro (dado como argumento) e acrescentando 
% um número romano ou algo parecido na continuação.

\begin{frame}[allowframebreaks]{Referências}
\bibliography{abntex2-modelo-references}
\end{frame}

% ----------------- FIM DO DOCUMENTO -----------------------------------------
\end{document}